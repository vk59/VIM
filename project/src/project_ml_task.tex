\documentclass[12pt,a4paper]{article}
\usepackage[utf8]{inputenc}
\usepackage[russian]{babel}
\usepackage[left=2.00cm, right=2.00cm, top=2.00cm, bottom=2.00cm]{geometry}
\linespread{1.25}
\usepackage{setspace}
\usepackage{indentfirst}
\setlength{\parindent}{1.25cm}
\let\paragraph\ignorespaces
\usepackage{tabularx}
\usepackage{multirow}
\usepackage{graphicx}
\usepackage{xcolor}
\usepackage{hyperref}


\begin{document}
	
\begin{titlepage}
	
\begin{center}
	\large Университет ИТМО\\[5cm]
	\LARGE Проект\\
	\normalsize по дисциплине <<Визуализация и моделирование>>\\[5cm]
\end{center}
\begin{flushright}
		\begin{minipage}{0.6\textwidth}
		\begin{flushleft}
			\large
			\singlespacing 
			\textbf{Автор:} Костылев Иван Михайлович\\
			\textbf{Поток:} 1.1\\
			\textbf{Группа:} K3240\\
			\textbf{Факультет:} ИКТ\\
			\textbf{Преподаватель:} Чернышева А.В.
		\end{flushleft}
	\end{minipage}
\end{flushright}

\vfill

\begin{center}
	{\large Санкт-Петербург, \the\year{ г.}}
\end{center}
 
\end{titlepage}
\normalsize


\large \textbf{Описание датасета}

\normalsize
	Датасет состоит из данных о студентах, их родителей и оценок, полученных ими по различным предметам. \\

Всего записей: 1000 \\


\large \textbf{Формальное описание}

\begin{tabular}{ | p{100pt} | p{100pt} | p{100pt} | p{40pt} | p{60pt} |}
\hline
Столбец & Описание & Значения & Формат & Шкала  \\ \hline
gender & пол студента & male / female & текст & Качеств номинальная \\ \hline
race/ethnicity & расовая классификация & group A / group B / group C / group D & текст & Качеств номинальная  \\ \hline
parental level of education & уровень образования родителей & collegue / school / bachelor's degree / others  & текст & Качеств номинальная  \\ \hline
lunch & оплата обеда & standart / free/reduced & текст & Качеств номинальная  \\ \hline
test preparation & подготовка к тесту & none / completed & текст & Качеств номинальная  \\ \hline
math score & оценка по математике & 0..100 & целое число & Колич относительная  \\ \hline
reading score & оценка по чтению & 0..100 & целое число & Колич относительная  \\ \hline
writting score & оценка по письму & 0..100 & целое число & Колич относительная  \\ \hline
\end{tabular}
\\
\\

\large \textbf{Задача машинного обучений}

Код с нормализацией, обучением модели и обработкой результатов приведен в Google Colab


\href{https://colab.research.google.com/drive/10O50LdP0oLxHryzGaBIZ97ue_vN9HoEK?usp=sharing}{$https://colab.research.google.com/drive/10O50LdP0oLxHryzGaBIZ97ue_vN9HoEK?usp=sharing$}


\textit{Данные разделим в сотношении 7:3 (обучение : тест)}

1. Ранее мы выяснили, что столбцы с данными корелируют друг с другом. Сейчас мы хотим \textbf{предсказать оценки студентов на основе личных данных студентов}

\textit{Результат: } 

коэффициент детерминации = 0.15604485344859867

MSE = 14.526800407137248

Такие результаты оказываются достаточно плохими. 



2. \textbf{Попробуем добавить к нашим данным данные об оценках двух предметов и попробуем предсказать третий}

\textit{Результат: } 

Получается коэффициент детерминации = 0.9435767981339643

MSE = 3.9813127576718337

Наша модель стала предсказывать результаты достаточно точно.



3. \textbf{Попробуем предсказать данные о студенте (например, пол) на основе его баллов с помощью k-NN классификатора}
\textit{Результат: } 

Наиболее точным классификатором оказался классификатор при k=4. 

\begin{verbatim}
              precision    recall  f1-score   support

           0       0.87      0.79      0.83       144
           1       0.82      0.89      0.86       156

    accuracy                           0.84       300
   macro avg       0.85      0.84      0.84       300
weighted avg       0.85      0.84      0.84       300
\end{verbatim}

Видно, что мы можем точно предсказать данные о поле студента лишь на основе оценок (без остальных параметров)

\newpage

\end{document}