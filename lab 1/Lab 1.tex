\documentclass[12pt,a4paper]{article}
\usepackage[utf8]{inputenc}
\usepackage[russian]{babel}
\usepackage[left=2.00cm, right=2.00cm, top=2.00cm, bottom=2.00cm]{geometry}
\linespread{1.25}
\usepackage{setspace}
\usepackage{indentfirst}
\setlength{\parindent}{1.25cm}
\let\paragraph\ignorespaces
\usepackage{tabularx}
\usepackage{multirow}
\usepackage{graphicx}



\begin{document}
	
\begin{titlepage}
	
\begin{center}
	\large Университет ИТМО\\[5cm]
	\LARGE Практическая работа №1\\
	\normalsize по дисциплине <<Визуализация и моделирование>>\\[5cm]
\end{center}
\begin{flushright}
		\begin{minipage}{0.6\textwidth}
		\begin{flushleft}
			\large
			\singlespacing 
			\textbf{Автор:} Костылев Иван Михайлович\\
			\textbf{Поток:} 1.1\\
			\textbf{Группа:} K3240\\
			\textbf{Факультет:} ИКТ\\
			\textbf{Преподаватель:} Чернышева А.В.
		\end{flushleft}
	\end{minipage}
\end{flushright}

\vfill

\begin{center}
	{\large Санкт-Петербург, \the\year{ г.}}
\end{center}
 
\end{titlepage}
\normalsize


\large \textbf{Описание датасета}

\normalsize
	Датасет состоит из данных о студентах, их родителей и оценок, полученных ими по различным предметам.



\large \textbf{Формальное описание}

\begin{tabular}{ | p{100pt} | p{140pt} | p{140pt} | p{50pt} | }
\hline
Столбец & Описание & Значения & Формат  \\ \hline
gender & пол студента & male / female & текст\\ \hline
race/ethnicity & расовая классификация & group A / group B / group C / group D & текст \\ \hline
parental level of education & уровень образования родителей & collegue / school / bachelor's degree / others & текст \\ \hline
lunch & оплата обеда & standart / free/reduced & текст \\ \hline
test preparation & подготовка к тесту & none / completed & текст \\ \hline
math score & оценка по математике & 0..100 & целое число \\ \hline
reading score & оценка по чтению & 0..100 & целое число \\ \hline
writting score & оценка по письму & 0..100 & целое число \\ \hline
\end{tabular}
\\


\large \textbf{Глобальные задачи}

1. Найти зависимость национальной группы от успеваемости студентов или показать, что такой нет;

2. Найти зависимость образования родителей от оценок студентов;

3. Ответить на вопрос "Помогает ли подготовка к тесту лучше сдать его?";

\textbf{ 4. На основе датасета предсказать среднюю оценку нового студента, который ещё не писал тесты;}

\textbf{ 5. Предсказать по новым данным оценки студента по всем предметам в отдельности.}

\end{document}